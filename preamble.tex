%\usepackage{arxiv}

%\usepackage[utf8]{inputenc} % allow utf-8 input
%\usepackage[T1]{fontenc}    % use 8-bit T1 fonts
\usepackage{hyperref}       % hyperlinks
\usepackage{url}            % simple URL typesetting
\usepackage{booktabs}       % professional-quality tables
\usepackage{amsmath}
\usepackage{amsfonts}       % blackboard math symbols
\usepackage{nicefrac}       % compact symbols for 1/2, etc.
\usepackage{microtype}      % microtypography
\usepackage{cleveref}       % smart cross-referencing
\usepackage{lipsum}         % Can be removed after putting your text content
\usepackage{graphicx}
\usepackage{natbib}
\usepackage{doi}

\usepackage{subfiles}
%\usepackage{algorithmicx}
%\usepackage[noend]{algpseudocode}  % skip EndFor etc
%\usepackage{algorithm}   % custom floats


%
% These are are recommended to typeset listings but not required. See the subsubsection on listing. Remove this block if you don't have listings in your paper.
\usepackage{newfloat}
\usepackage{listings}

\usepackage[normalem]{ulem} % strikeout, remove before submission
\usepackage[]{todonotes}
\setlength{\marginparwidth}{2cm} % for todonotes only
\DeclareSymbolFont{bbold}{U}{bbold}{m}{n}  % blackboard bold 1
\DeclareSymbolFontAlphabet{\mathbbold}{bbold}% blackboard bold 1

% \usepackage{bbm}
% \usepackage{bbold}
\usepackage{mathrsfs}       % mathscr
\usepackage{bm} 
\usepackage{latexsym}
\usepackage{amssymb}
\usepackage{caption}
\usepackage{subcaption}


\usepackage{tikz}
\usetikzlibrary{shapes,arrows}
\usetikzlibrary{arrows,calc,positioning, automata,decorations.pathreplacing,quotes}
\usetikzlibrary{bayesnet}


\DeclareCaptionStyle{ruled}{labelfont=normalfont,labelsep=colon,strut=off} % DO NOT CHANGE THIS
\lstset{%
	basicstyle={\footnotesize\ttfamily},% footnotesize acceptable for monospace
	numbers=left,numberstyle=\footnotesize,xleftmargin=2em,% show line numbers, remove this entire line if you don't want the numbers.
	aboveskip=0pt,belowskip=0pt,%
	showstringspaces=false,tabsize=2,breaklines=true}
%\floatstyle{ruled}
%\newfloat{listing}{tb}{lst}{}
%\floatname{listing}{Listing}
%
% Keep the \pdfinfo as shown here. There's no need
% for you to add the /Title and /Author tags.
\pdfinfo{
/TemplateVersion (2023.1)
}

%\setcounter{secnumdepth}{2} %May be changed to 1 or 2 if section numbers are desired.

% The file aaai23.sty is the style file for AAAI Press
% proceedings, working notes, and technical reports.
%
\newcommand\nb[1]{\todo{#1}}
% \newcommand\nb[1]{}
\newcommand\numberthis{\addtocounter{equation}{1}\tag{\theequation}}
\newcommand{\tm}{t{\operatorname{-}}1} % "t-1" but the minus sign is too big in the subscript
\newcommand{\tmm}{t{\operatorname{-}}2} % "t-2" but the minus sign is too big in the subscript
\newcommand{\tp}{t{\operatorname{+}}1} % "t+1" but the minus sign is too big in the subscript
\newcommand{\Ex}{\mathbb{E}}
\newcommand{\var}{\operatorname{Var}}
\newcommand{\cov}{\operatorname{Cov}}
\newcommand{\corr}{\operatorname{Corr}}
\newcommand{\argmin}{\operatorname{arg min}}
\newcommand{\argmax}{\operatorname{arg max}}
\newcommand{\vecop}{\operatorname{vec}}
\newcommand{\dd}{\mathrm{d}}
\newcommand{\pd}{\mathrm{d}}
\newcommand{\bb}[1]{\mathbb{#1}}
\newcommand{\vv}[1]{\boldsymbol{#1}}
\newcommand{\mm}[1]{\mathrm{#1}}
\newcommand{\rv}[1]{\mathsf{#1}}
\newcommand{\vrv}[1]{\vv{\rv{#1}}}
\newcommand{\dist}[1]{\mathcal{#1}}
\newcommand{\set}[1]{\mathcal{#1}}
\newcommand{\op}[1]{\mathscr{#1}}
\newcommand{\proj}{\mm{P}}
\newcommand{\disteq}{\stackrel{\mathrm{d}}{=}}
\newcommand{\approxdisteq}{\stackrel{\mathrm{d}}{\approx}}
\newcommand{\gvn}{\mid}
\renewcommand{\Pr}{\mathbb{P}}
\newcommand{\solop}{\mathcal{G}^{\dagger}}
% \newcommand{\lat}{\vrv{b}}   % Kept changing the symbol for latent state
% \newcommand{\latst}{b}       % so now it is a macro
% \newcommand{\latsp}{\mathcal{B}}
\newcommand{\forcing}{\vrv{u}}   
% \newcommand{\forcing}{\tilde{\forcing}}   
\newcommand{\forcingst}{\vv{u}}
% \newcommand{\forcingst}{\tilde{\forcingst}} % discretized state space 
\newcommand{\forcingsp}{\mathcal{U}}
\newcommand{\state}{\vrv{z}}   
% \newcommand{\state}{\tilde{\state}}   % discretized state rv
\newcommand{\statest}{\vv{z}}
% \newcommand{\statest}{\tilde{\statest}}    % discretized state space
\newcommand{\statesp}{\mathcal{Z}}
\newcommand{\outp}{\vrv{y}}   
\newcommand{\outpst}{\vv{y}}      
\newcommand{\outpsp}{\mathcal{Y}}
\newcommand{\latwt}{\vrv{w}}
\newcommand{\latwtst}{\vv{w}}
\newcommand{\param}{\vrv{\theta}}  % misc param
\newcommand{\paramst}{\vv{\theta}}
\newcommand{\seenstate}{\proj_{\state}}
\newcommand{\seeninp}{\proj_{\forcing}}
\newcommand{\netparamsp}{\boldsymbol{\Theta}}
\newcommand{\netparamst}{\boldsymbol{\theta}}
\newcommand{\Law}{\operatorname{Law}}
\newcommand\indep{\protect\mathpalette{\protect\independenT}{\perp}} % indep symbol
\def\independenT#1#2{\mathrel{\rlap{$#1#2$}\mkern2mu{#1#2}}}
% \newcommand{\one}{\mathbb{1}}
\newcommand{\one}{\mathbbold{1} }
\newcommand{\meth}{{\sc Metho}}
\DeclareRobustCommand{\russ}[1]{\textcolor{blue}{{#1}}}
\DeclareRobustCommand{\danm}[1]{\textcolor{orange}{{#1}}}

% \newcommand{\hatlatents}{\widehat{\lat}_{\boldsymbol{\kappa}}}

% Title

% Your title must be in mixed case, not sentence case.
% That means all verbs (including short verbs like be, is, using,and go),
% nouns, adverbs, adjectives should be capitalized, including both words in hyphenated terms, while
% articles, conjunctions, and prepositions are lower case unless they
% directly follow a colon or long dash
