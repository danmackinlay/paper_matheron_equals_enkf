\section{Introduction}
This paper introduces a method for inversion of models of dynamical systems whose state and static parameters are both very high-dimensional and highly correlated, but which may be modelled by a physic simulator. 
Problems of this sort arise in geophysics, hydrology, and medical imaging, and are central to many problems in machine learning.
In fact, inferring any hidden cause through its influence on some dynamical behaviour may be understood as a problem of this type.
Traditionally, the methods to solve inverse problems have been statistical \cite{TarantolaInverse2005} and have often relied on strategies like model emulation / surrogate modelling and dimension reduction to make these feasible in applied settings.  

 We propose the MaTHeron Ensemble inversion approach (\meth{}), which efficiently utilises a (possibly black-box) forward operator to sample from a posterior distribution over unobserved  static parameters.
\meth{} advances our capability to handle such models, by treating the observations as an  approximation to Gaussian variates utilising the the ensemble methods of the data assimilation community~\cite{EvensenData2009}.
Gaussian variates may be updated, sample-wise, using the Matheron update~\cite{DoucetNote2010,WilsonEfficiently2020}, without the need to calculate covariances explicitly.
The method is general enough to be applicable to any dynamic model that is well-approximated by assuming linearity in expectation, scales to very large systems, and requires few assumptions.
% The prior distributions may be supplied by any model from which we can sample, and are not required to, for example, possess stationary or homogenaeity.

The method is structurally similar to the Ensemble Kalman filter, in that all updates between random variates are conducted via ensembles of samples rather than explicit densities.
However, \meth{} exploits the closed form Matheron updates for static parameters rather than states. This generates closed-form ensemble update rules which are capable of assimilating multiple observations without ever constructing the prior covariance matrix, and thus scales to very high dimensions.