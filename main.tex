%%%%%%%% ICML 2023 EXAMPLE LATEX SUBMISSION FILE %%%%%%%%%%%%%%%%%

\documentclass{article}
% !TEX root = ./main.tex

% \usepackage{hyperref}       % hyperlinks
\usepackage{url}            % URL typesetting
\usepackage{booktabs}       % professional-quality tables
\usepackage{amsmath, amsfonts, amssymb} % math packages
\usepackage{mathtools}      % extended math tools
\usepackage{nicefrac}       % compact fractions
\usepackage{microtype}      % microtypography
\usepackage{cleveref}       % intelligent cross-referencing
\usepackage{graphicx}       % graphics
\usepackage{natbib}         % bibliography support
\usepackage{doi}            % DOI formatting
\usepackage{subfiles}       % subfile support
\usepackage{listings}       % source code listings
\usepackage[normalem]{ulem} % underline/strikeout (remove before final submission)
\usepackage[]{todonotes}    % inline todo notes (optional)
\usepackage{orcidlink}
\usepackage{xparse}

% % TikZ for diagrams (if needed)
% \usepackage{tikz}
% \usetikzlibrary{shapes,arrows,calc,positioning,automata,decorations.pathreplacing,quotes}
% \usetikzlibrary{bayesnet}

% Custom commands and notation
\newcommand{\Ex}{\mathbb{E}}
\newcommand{\var}{\operatorname{Var}}
\newcommand{\cov}{\operatorname{Cov}}
\newcommand{\corr}{\operatorname{Corr}}
\newcommand{\argmin}{\operatorname{arg\,min}}
\newcommand{\argmax}{\operatorname{arg\,max}}
\newcommand{\diag}{\operatorname{diag}}
\newcommand{\vecop}{\operatorname{vec}}
\newcommand{\dd}{\mathrm{d}}
\newcommand{\pd}{\mathrm{d}}
\newcommand{\bb}[1]{\mathbb{#1}}
\newcommand{\vv}[1]{\boldsymbol{#1}}
\newcommand{\mm}[1]{\mathrm{#1}}
\newcommand{\mmmean}[1]{\bar{\mathrm{#1}}}
\newcommand{\mmdev}[1]{\breve{\mathrm{#1}}}
\newcommand{\rv}[1]{\mathsf{#1}}
\newcommand{\vrv}[1]{\vv{\rv{#1}}}
\newcommand{\dist}[1]{\mathcal{#1}}
% \newcommand{\set}[1]{\mathcal{#1}}
\newcommand{\op}[1]{\mathcal{#1}}
\newcommand{\proj}{\mm{P}}
\newcommand{\disteq}{\stackrel{\mathrm{d}}{=}}
\newcommand{\Normal}{\mathcal{N}}
\newcommand{\approxdisteq}{\stackrel{\mathrm{d}}{\approx}}
\newcommand{\gvn}{\mid}
\renewcommand{\Pr}{\mathbb{P}}
\newcommand{\solop}{\mathcal{G}^{\dagger}}
%
% Measure notation commands:

\NewDocumentCommand{\Law}{o}{%
  \mu\IfValueT{#1}{_{#1}}%
}

\NewDocumentCommand{\ELaw}{o}{%
  \widehat{\mu}\IfValueT{#1}{_{#1}}%
}
% Independence symbol
\newcommand\indep{\protect\mathpalette{\protect\independenT}{\perp}}
\def\independenT#1#2{\mathrel{\rlap{$#1#2$}\mkern2mu{#1#2}}}

\newcommand{\one}{\mathbbold{1}}

% \theoremstyle{plain}
% \newtheorem{theorem}{Theorem}[section]
% \newtheorem{lemma}[theorem]{Lemma}
% \newtheorem{definition}[theorem]{Definition}
% \newtheorem{proposition}[theorem]{Proposition}

% Recommended, but optional, packages for figures and better typesetting:
\usepackage{microtype}
\usepackage{graphicx}
%\usepackage{subfigure}
\usepackage{booktabs} % for professional tables

% hyperref makes hyperlinks in the resulting PDF.
% If your build breaks (sometimes temporarily if a hyperlink spans a page)
% please comment out the following usepackage line and replace
% \usepackage{icml2023} with \usepackage[nohyperref]{icml2023} above.
\usepackage{hyperref}


% Attempt to make hyperref and algorithmic work together better:
\newcommand{\theHalgorithm}{\arabic{algorithm}}

% Use the following line for the initial blind version submitted for review:
\usepackage{icml2023}

% If accepted, instead use the following line for the camera-ready submission:
% \usepackage[accepted]{icml2023}

% For theorems and such
\usepackage{amsmath}
\usepackage{amssymb}
\usepackage{mathtools}
\usepackage{amsthm}

% if you use cleveref..
%\usepackage[capitalize,noabbrev]{cleveref}

%%%%%%%%%%%%%%%%%%%%%%%%%%%%%%%%
% THEOREMS
%%%%%%%%%%%%%%%%%%%%%%%%%%%%%%%%
\theoremstyle{plain}
\newtheorem{theorem}{Theorem}[section]
\newtheorem{proposition}[theorem]{Proposition}
\newtheorem{lemma}[theorem]{Lemma}
\newtheorem{corollary}[theorem]{Corollary}
\theoremstyle{definition}
\newtheorem{definition}[theorem]{Definition}
\newtheorem{assumption}[theorem]{Assumption}
\theoremstyle{remark}
\newtheorem{remark}[theorem]{Remark}

% Todonotes is useful during development; simply uncomment the next line
%    and comment out the line below the next line to turn off comments
%\usepackage[disable,textsize=tiny]{todonotes}
%\usepackage[textsize=tiny]{todonotes}


% The \icmltitle you define below is probably too long as a header.
% Therefore, a short form for the running title is supplied here:



\icmltitlerunning{The Ensemble Kalman Update as Matheron update}

% Here you can change the date presented in the paper title
%\date{September 9, 1985}
% Or remove it
%\date{}

%\author{ \href{https://orcid.org/0000-0001-6077-2684}{\includegraphics[scale=0.06]{orcid.pdf}\hspace{1mm}Dan B MacKinlay}%\thanks{Use footnote for providing further
		%information about author (webpage, alternative
		%address)---\emph{not} for acknowledging funding agencies.}
%  \\\
%	CSIRO's Data61 \\
%	\texttt{Dan.MacKinlay@data61.csiro.au} \\
%	%% examples of more authors
%	\And
%	Dan Pagendam \\
%		CSIRO's Data61 \\
%	\AND
%	Russell Tsuchida \\
%	CSIRO's Data61 \\
%	\And
%	Petra Kuhnert \\
%	CSIRO's Data61 \\
%	\And
%	   Sreekanth Janardhanan \\
%	CSIRO \\
%}


\begin{document}

\twocolumn[
\icmltitle{The Ensemble Kalman Update is an empirical Matheron update}

% It is OKAY to include author information, even for blind
% submissions: the style file will automatically remove it for you
% unless you've provided the [accepted] option to the icml2023
% package.

% List of affiliations: The first argument should be a (short)
% identifier you will use later to specify author affiliations
% Academic affiliations should list Department, University, City, Region, Country
% Industry affiliations should list Company, City, Region, Country

% You can specify symbols, otherwise they are numbered in order.
% Ideally, you should not use this facility. Affiliations will be numbered
% in order of appearance and this is the preferred way.
\icmlsetsymbol{equal}{*}



\begin{icmlauthorlist}
\icmlauthor{Firstname1 Lastname1}{equal,yyy}
\icmlauthor{Firstname2 Lastname2}{equal,yyy,comp}
\icmlauthor{Firstname3 Lastname3}{comp}
\icmlauthor{Firstname4 Lastname4}{sch}
\icmlauthor{Firstname5 Lastname5}{yyy}
\icmlauthor{Firstname6 Lastname6}{sch,yyy,comp}
\icmlauthor{Firstname7 Lastname7}{comp}
%\icmlauthor{}{sch}
\icmlauthor{Firstname8 Lastname8}{sch}
\icmlauthor{Firstname8 Lastname8}{yyy,comp}
%\icmlauthor{}{sch}
%\icmlauthor{}{sch}
\end{icmlauthorlist}

\icmlaffiliation{yyy}{Department of XXX, University of YYY, Location, Country}
\icmlaffiliation{comp}{Company Name, Location, Country}
\icmlaffiliation{sch}{School of ZZZ, Institute of WWW, Location, Country}

\icmlcorrespondingauthor{Firstname1 Lastname1}{first1.last1@xxx.edu}
\icmlcorrespondingauthor{Firstname2 Lastname2}{first2.last2@www.uk}

% You may provide any keywords that you
% find helpful for describing your paper; these are used to populate
% the "keywords" metadata in the PDF but will not be shown in the document
\icmlkeywords{likelihood free, ensemble, Gaussian Process, inverse modelling, physics}

\vskip 0.3in
]

% this must go after the closing bracket ] following \twocolumn[ ...

% This command actually creates the footnote in the first column
% listing the affiliations and the copyright notice.
% The command takes one argument, which is text to display at the start of the footnote.
% The \icmlEqualContribution command is standard text for equal contribution.
% Remove it (just {}) if you do not need this facility.

\printAffiliationsAndNotice{}  % leave blank if no need to mention equal contribution
%\printAffiliationsAndNotice{\icmlEqualContribution} % otherwise use the standard text.

\begin{abstract}
    A popular means of 
    
    The cost of the resulting algorithm scales linearly with the dimension of the estimands and  does not require the calculation or storage of large covariance matrices.
    %Nor does it require artificial evolution dynamics to be applied to static latent parameters.
    We do not require exact likelihood evaluations of the prior or posterior samples.
    However, we do require the ability to simulate from the prior, and sufficient regularity of the forward operator.
    Our method achieves orders-of-magnitude speed-up over a classical physical model inversion approach for high-dimensional models, while maintaining acceptable accuracy. This increases the dimensionality of the estimand at which inference is feasible, without resorting to dimension reduction methods.
    % \russ{Abstracts should be 4-6 sentences and a single paragraph}
    \footnote{\textcolor{blue}{Link to publically available code to be provided upon acceptance. Reviewers, see attached .zip.}}
\end{abstract}

\subfile{sections/01_introduction}
\subfile{sections/02_preliminaries}
\subfile{sections/03_problemsetting}
\subfile{sections/04_experiments}
\subfile{sections/05_conclusion}

\clearpage
\bibliographystyle{icml2023}
\bibliography{refs.bib}

%%% Uncomment this line and comment out the ``thebibliography'' section below to use the external .bib file (using bibtex) .


%%% Uncomment this section and comment out the \bibliography{references} line above to use inline references.
% \begin{thebibliography}{1}

% 	\bibitem{kour2014real}
% 	George Kour and Raid Saabne.
% 	\newblock Real-time segmentation of on-line handwritten arabic script.
% 	\newblock In {\em Frontiers in Handwriting Recognition (ICFHR), 2014 14th
% 			International Conference on}, pages 417--422. IEEE, 2014.

% 	\bibitem{kour2014fast}
% 	George Kour and Raid Saabne.
% 	\newblock Fast classification of handwritten on-line arabic characters.
% 	\newblock In {\em Soft Computing and Pattern Recognition (SoCPaR), 2014 6th
% 			International Conference of}, pages 312--318. IEEE, 2014.

% 	\bibitem{hadash2018estimate}
% 	Guy Hadash, Einat Kermany, Boaz Carmeli, Ofer Lavi, George Kour, and Alon
% 	Jacovi.
% 	\newblock Estimate and replace: A novel approach to integrating deep neural
% 	networks with existing applications.
% 	\newblock {\em arXiv preprint arXiv:1804.09028}, 2018.

% \end{thebibliography}

\subfile{sections/appendix}


\iffalse
\clearpage

\section{TODO}\label{sec:todo}
\begin{itemize}
    \item \sout{text: write up competing inversion methods, POD etc}
    \item \sout{text: connect to Matheron updates}
    \item \sout{code: EKF baseline?}
    \item \sout{code: MCMC baseline?} out of scope
    \item plot: EKF baseline
    \item code: GD baseline
    \item \sout{text: remove GD baseline; there is no time for that}
    \item text: hyperparameters, how to select (WIP)
    \item \sout{text: EKF stochastic forward opp. How does that even work?} It doesn't work
    \item \sout{plot: hyperparam effect}
    \item text: definitive \(\mathcal{O}\) costing, in ensemble size, in simulation runs, in problem dimension
    \item text: Acknowledge Rui's proofreading
    \item text: ICML style 
    \begin{itemize}
        \item author formatting
        \item figure resizing
    \end{itemize}
    \item code: alternate sim model
    \begin{itemize}
        \item \sout{Lorentz 96?}
        \item non-smooth prior
        \item \sout{how about a high-dimensional non-linear VARIMA with a latent forcing?}
        \item \sout{source localization}
        \item advection-diffusion?
    \end{itemize}
    \item text: write up algorithm
    \item \sout{code: deterministic experiments}
    \item plot: an alternate sim model
    \item \sout{text: EGU submission abstract}
    \item \sout{code: Switch to generic MVP inversion via \url{https://github.com/cornellius-gp/linear_operator}}
    \item \sout{text: can we relate nonlinear Matheron to a probability divergence? I think it is something like ensemble expectation propagation, which implies approx reverse-KL}
    \item \sout{text: should we introduce ensemble belief propagation via matheron updates?}
    \item \sout{text: KVSD comparison?}
    \item \sout{plot: switch to \url{https://tueplots.readthedocs.io/en/latest/} to get consistent fonts, figure sizes}
    \item \sout{code: advection operator is applied twice in forward step. Does this mean that we have lost interpretation of the viscosity? Check} Irrelevant for now
    \item \sout{text: can the prior have hyperparameters as well?} out of scope
\end{itemize}
\fi
\end{document}
