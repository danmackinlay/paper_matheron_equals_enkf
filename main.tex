\documentclass{article}

\usepackage{amsmath, amssymb, mathtools, amsthm}
\usepackage{graphicx}
\usepackage{hyperref}
\usepackage{icml2023}

% THEOREMS
\theoremstyle{plain}
\newtheorem{theorem}{Theorem}
\newtheorem{lemma}[theorem]{Lemma}

\icmltitlerunning{The Ensemble Kalman Update as an Empirical Matheron Update}

\begin{document}

\twocolumn[
\icmltitle{The Ensemble Kalman Update as an Empirical Matheron Update}

\icmlsetsymbol{equal}{*}

\begin{icmlauthorlist}
\icmlauthor{Firstname1 Lastname1}{equal,yyy}
\icmlauthor{Firstname2 Lastname2}{equal,yyy,comp}
% Additional authors omitted for brevity
\end{icmlauthorlist}

\icmlaffiliation{yyy}{Department of XXX, University of YYY, Location, Country}
\icmlaffiliation{comp}{Company Name, Location, Country}

\icmlcorrespondingauthor{Firstname1 Lastname1}{first1.last1@xxx.edu}

\vskip 0.3in
]

\printAffiliationsAndNotice{}

\begin{abstract}
We show that the ensemble update in the Ensemble Kalman Filter (EnKF) can be interpreted as an empirical Matheron update for Gaussian conditioning. This connection provides a new perspective on the EnKF and suggests potential improvements by leveraging properties of the Matheron update.
\end{abstract}

\section{Introduction}

The Ensemble Kalman Filter (EnKF) is a widely used method for state estimation and data assimilation in high-dimensional systems, particularly in geosciences and meteorology \cite{EvensenData2009}. Despite its popularity, the theoretical understanding of the EnKF remains an active area of research.

In this paper, we provide a new perspective on the EnKF by showing that the ensemble update step is equivalent to an empirical Matheron update. The Matheron update is a pathwise conditioning method for Gaussian random variables \cite{Matheron1962,DoucetNote2010,WilsonPathwise2021}. By interpreting the EnKF update in terms of the Matheron update, we gain new insights into its properties and potential areas for improvement.

\section{Preliminaries}

\subsection{Gaussian Conditioning and the Matheron Update}

Let $\mathbf{y}$ and $\mathbf{w}$ be jointly Gaussian random variables with
\begin{align}
\begin{bmatrix} \mathbf{y} \\ \mathbf{w} \end{bmatrix} \sim \mathcal{N} \left( \begin{bmatrix} \mathbf{m}_{\mathbf{y}} \\ \mathbf{m}_{\mathbf{w}} \end{bmatrix}, \begin{bmatrix} \mathbf{K}_{\mathbf{y}\mathbf{y}} & \mathbf{K}_{\mathbf{y}\mathbf{w}} \\ \mathbf{K}_{\mathbf{w}\mathbf{y}} & \mathbf{K}_{\mathbf{w}\mathbf{w}} \end{bmatrix} \right).
\end{align}

The conditional distribution of $\mathbf{y}$ given $\mathbf{w} = \mathbf{w}^\ast$ is Gaussian with mean and covariance:
\begin{align}
\hat{\mathbf{m}} &= \mathbf{m}_{\mathbf{y}} + \mathbf{K}_{\mathbf{y}\mathbf{w}} \mathbf{K}_{\mathbf{w}\mathbf{w}}^{-1} (\mathbf{w}^\ast - \mathbf{m}_{\mathbf{w}}), \label{eq:cond_mean} \\
\hat{\mathbf{K}} &= \mathbf{K}_{\mathbf{y}\mathbf{y}} - \mathbf{K}_{\mathbf{y}\mathbf{w}} \mathbf{K}_{\mathbf{w}\mathbf{w}}^{-1} \mathbf{K}_{\mathbf{w}\mathbf{y}}. \label{eq:cond_cov}
\end{align}

An alternative way to sample from the conditional distribution is via the \emph{Matheron update} \cite{DoucetNote2010}:
\begin{align}
\mathbf{y} \mid \mathbf{w} = \mathbf{w}^\ast \stackrel{d}{=} \mathbf{y} + \mathbf{K}_{\mathbf{y}\mathbf{w}} \mathbf{K}_{\mathbf{w}\mathbf{w}}^{-1} (\mathbf{w}^\ast - \mathbf{w}). \label{eq:matheron_update}
\end{align}

Here, $\mathbf{y}$ and $\mathbf{w}$ on the right-hand side are jointly distributed as in the prior, and the update adjusts $\mathbf{y}$ to be consistent with the observation $\mathbf{w}^\ast$.

\subsection{Ensemble Kalman Filter}

The EnKF uses an ensemble of particles to represent the state distribution and updates the ensemble members using Kalman filter equations \cite{EvensenData2009}. For each ensemble member, the update is:
\begin{align}
\mathbf{y}^{(i)} \leftarrow \mathbf{y}^{(i)} + \mathbf{K}_{\mathbf{y}\mathbf{w}} \mathbf{K}_{\mathbf{w}\mathbf{w}}^{-1} (\mathbf{w}^\ast - \mathbf{w}^{(i)}), \label{eq:enkf_update}
\end{align}
where $\mathbf{y}^{(i)}$ and $\mathbf{w}^{(i)}$ are the $i$-th ensemble members of the state and observation variables, respectively.

\section{Ensemble Update as Empirical Matheron Update}

We now show that the ensemble update in the EnKF is an empirical version of the Matheron update.

\subsection{Empirical Covariances}

In the EnKF, the sample covariances are used to approximate the true covariances:
\begin{align}
\mathbf{K}_{\mathbf{y}\mathbf{w}} &\approx \frac{1}{N - 1} \sum_{i=1}^N (\mathbf{y}^{(i)} - \bar{\mathbf{y}}) (\mathbf{w}^{(i)} - \bar{\mathbf{w}})^\top, \\
\mathbf{K}_{\mathbf{w}\mathbf{w}} &\approx \frac{1}{N - 1} \sum_{i=1}^N (\mathbf{w}^{(i)} - \bar{\mathbf{w}}) (\mathbf{w}^{(i)} - \bar{\mathbf{w}})^\top,
\end{align}
where $\bar{\mathbf{y}}$ and $\bar{\mathbf{w}}$ are the sample means:
\begin{align}
\bar{\mathbf{y}} = \frac{1}{N} \sum_{i=1}^N \mathbf{y}^{(i)}, \quad \bar{\mathbf{w}} = \frac{1}{N} \sum_{i=1}^N \mathbf{w}^{(i)}.
\end{align}

\subsection{Ensemble Update Derivation}

Substituting the empirical covariances into the Kalman gain and the update equation \eqref{eq:enkf_update}, we have:
\begin{align}
\mathbf{y}^{(i)} &\leftarrow \mathbf{y}^{(i)} + \mathbf{C}_{\mathbf{y}\mathbf{w}} \mathbf{C}_{\mathbf{w}\mathbf{w}}^{-1} (\mathbf{w}^\ast - \mathbf{w}^{(i)}), \label{eq:ensemble_update}
\end{align}
where $\mathbf{C}_{\mathbf{y}\mathbf{w}}$ and $\mathbf{C}_{\mathbf{w}\mathbf{w}}$ are the empirical covariances computed from the ensemble.

This update is equivalent to applying the Matheron update \eqref{eq:matheron_update} to each ensemble member using the empirical covariances.

\subsection{Interpretation}

The Matheron update adjusts the prior samples $\mathbf{y}$ to obtain samples from the conditional distribution given the observation $\mathbf{w}^\ast$. In the EnKF, the ensemble update adjusts each ensemble member in a way that reflects this conditioning, using the empirical covariances.

\section{Implications and Discussion}

\subsection{Understanding the EnKF}

Interpreting the EnKF update as an empirical Matheron update explains why the EnKF works well even in nonlinear and non-Gaussian settings. The Matheron update relies on sample paths rather than explicit distributional assumptions, making it robust to deviations from Gaussianity.

\subsection{Potential Improvements}

This connection suggests that improvements in sampling from conditional distributions, such as using better estimates of covariances or incorporating higher-order statistics, could enhance the performance of the EnKF.

\section{Conclusion}

We have shown that the ensemble update in the EnKF is equivalent to an empirical Matheron update. This insight provides a new perspective on the EnKF and opens avenues for further research and potential improvements in ensemble-based data assimilation methods.

\bibliographystyle{icml2023}
\bibliography{refs}

\end{document}
